
\documentclass{article}
\usepackage[utf8]{inputenc}
\usepackage{amsmath,amssymb,amsthm,tabu,enumerate,tikz}
\usepackage[margin=1in]{geometry}
\usepackage{verbatim} % Allows Multi-line comments 
\usepackage{multicol}
\usetikzlibrary{automata,positioning}
\newcommand{\encode}[1]{\langle #1 \rangle}
\usepackage{paralist}
  \let\enumerate\compactenum
\title{Midterm Solutions}
\author{dcobbley }
\date{March 2015}
\begin{document}
\maketitle
------------------------------------------------------------------------
CS 321 Languages and Compiler Design I, Winter 2015

Assignment 4: "Static"
------------------------------------------------------------------------

This is the fourth and final homework assignment for CS 321.

Submission Procedure:
---------------------
Submissions to this assignment are formally due at noon on Monday,
March 16.  This is the same day as the final exam.  That said, the
dropbox for submissions will remain open until noon on Wednesday,
March 18, and submissions received by that later time will be accepted
without penalty.  I'm making this arrangement because:

- I really want you to focus on being done by (or, ideally, before
  March 16).  In particular, I expect to be busy grading final exam
  papers in the days immediately after March 16, so I likely won't
  be around to give much help with last minute homework questions.

- I want to give you a little flexibility, particularly so that you
  are able to find time to prepare for the final (which, incidentally,
  is worth TWICE as many points in the overall grading scheme as this
  assignment).

- I want to emphasize that noon on March 18 really is the absolute
  latest time that I can accept work for grading.  I may not be able
  to meet the university's grading deadlines for anything that is sent
  to me after that.

To submit your solution to this assignment, you will be expected to
create a zip file using a command of the following form:

  zip -r9 YOURNAMEGOESHERE.zip ast/*.java extras.txt

The "extras.txt" part on the end is only needed if you have attempted
the extra credit portion of this assignment that is described at the
end of this document.  Once you have created the zip file, you should
upload it to the dropbox on D2L.  You should not include any additional
materials.

Time Management, and Asking for Help:
-------------------------------------
This exercise is expected to take a total of between 4 and 12 hours
to complete, depending on your level of programming experience, and
including approximately one hour of focused time to read this
document.  Add extra time to this if you haven't already worked
through the recent slides and lab materials.

Note that you may not be able to finish the full assignment in this
time; the best strategy is to start early, and to be ready to ask
questions if you get stuck or need clarification.

Of course, there is much to be learned by trying to solve problems
like this on your own.  However, given the timing constraints, you
simply don't have the luxury of being able to spend, say, four hours
trying to figure out some small detail by yourself.  Instead, if you
find yourself struggling with something for, say an hour, then stop,
and spend some time instead to formulate a question, as precisely as
you can, that describes the problem you're having.  Maybe it's
something in the text that doesn't make sense?  Or maybe you know
what you want the program to do, but you don't know how to express
it?  Once that's done, post your question on D2L, or come talk to the
TA or the instructor.  We'll do our best to get you pointed in the
right direction as quickly as we can, and you will have spent a lot
less on this than the four hours it might have taken on your own.

Please do not feel bad about asking for help.  After all, that is the
way that the real world works: we partner with colleagues in a team,
or interact with our customers to get a better understanding about
what they need.  And of course, we all frequently reach out to online
resources like Google for help.  From that perspective, asking the TA
or instructor for some assistance really isn't so very different.
(And in case you were worrying about it: I promise that we won't
count asking for help against you when it comes to grading!)

While you can contact the TA and/or instructor directly, I encourage
you to post questions on D2L whenever possible, so long as that
doesn't involve sharing solutions.  Because D2L reaches a larger
audience, I prioritize responding to postings there over responding
to individual messages.  In addition, if you post to D2L, then you
may well get a faster reply, perhaps from a friendly classmate, for
example, who had struggled through a similar issue.  In fact, I am
considering options to award some small amount of extra credit to
students who help other students on D2L (so long, again, as that
doesn't involve sharing details of solutions).

Now it's time for the technical material ...

------------------------------------------------------------------------
Building a Static Analyzer:
---------------------------
With the successful conclusion of Homework 3, you have built a parser
that can read source text and generate corresponding abstract syntax
trees for a simple, but realistic programming language.  The parser
is able to determine if the program contains any syntax errors, but
of course there are common programming errors that it is not able to
detect.  The following (yes, rather lengthy) list describes some of
the additional properties that we would expect to hold in valid
programs.  It will be the job of the static analysis components of
the compiler to check that these properties are satisfied.  And it
will be your job in this assignment to complete the implementation of
that static analysis component!


READ HERE

EXAMPLES OF STATIC ANALYSIS PROPERTIES:
---------------------------------------
Each of the following property descriptions includes a label, written
in CamelCase.  These labels will be used throughout this assignment
to reference individual properties within this list.

MainDefined
  Every program should contain a definition of a function called
  main.

MainVoid
  The main function should have void return type.

MainNoParameters
  The main function should not have any formal parameters.

FunctionNamesUnique
  A program cannot contain two functions with the same name.  [But
  it is permitted for a function to have the same name as a global
  variable defined elsewhere in the same program: we say that "the
  namespaces for functions and globals are disjoint".]

FormalsUnique
  The list of formal parameters in a function declaration cannot
  contain two variables with the same name.

GlobalsUnique
  A program cannot contain two global variables with the same
  name.

GlobalsInitialized
  Every variable introduced in a global variable declaration must
  specify an initializer expression.  [This condition could have
  been enforced by the parser, but we will implement it as part of
  static analysis instead.]

GlobalsNoCalls
  The initializer expression in a global variable declaration must
  not include any function calls.  [We want to avoid situations in
  which the called function could access a global variable that
  has not yet been initialized.]

IfBoolean
  The test expression in an "if" statement must have type boolean.

WhileBoolean
  The test expression in a "while" statement must have type
  boolean.

PrintArgument
  The expression in a print statement can have any type (int,
  double, or boolean).

ReturnValueRequired
  Return statements of the form "return;" (i.e., without an
  expression specifying a return value) are not allowed in the
  body of functions that have a non-void return type.

ReturnVoidRequired
  Return statements of the form "return e;" (i.e., with an
  expression specifying a return value) are not allowed in the
  body of functions that have a void return type.

ReturnType
  If a "return e;" statement appears in the body of a function
  with a non-void return type, then the type of e must match the
  return type of the function.  Note, however, that it is
  permitted to return a value of type int in a function
  with return type double, or to return a value of type double
  in a function with return type double.

InitVarEntryType
  The type of the expression in an initializer in either a local
  or global declaration should match the type of the variable that
  is being defined.  As in previous cases, however, an expression
  of type int or double may be used in places where an expression
  of double or int, respectively, is expected.  In these cases,
  the expectation is that the value produced by the initializer
  expression will be cast to match the type of the variable that
  is being initialized.

LocalsUnique
  A local variable declaration cannot contain two variables with
  the same name.  [But it is permitted for a local variable
  declaration to use a variable name that has previously been
  introduced in the same function, either as a parameter name or
  in a separate local variable declaration, effectively hiding the
  earlier definition of that name.]

VarDefined
  Any variable name that is used in a program, including any use
  on the left hand side of an assignment, must have an associated
  definition, either in a preceding local declaration or parameter
  list, or in a global declaration.  Any variable that is
  mentioned in the initializer portion of a global variable
  declaration must have been defined in an earlier global variable
  declaration.  (Either earlier in the same list of variable
  introductions, or in a separate globals declaration earlier in
  the program.)

  [Note that it is permitted to reference a global variable in the
  body of a function, even if the declaration of that global
  variable appears after the definition of the function in the
  source program.]

ArithBinArgsNumeric
  The arguments of an arithmetic operator (+, -, *, or /) must
  both be of numeric type (int or double).  If the arguments are
  of the same type, then the result will also have that same type.
  If one of the arguments has type int and the other has type
  double, then the return type will be double.

RelBinArgs:
  The relational operators ($<$, $<$=, $>$, $>$=, ==, or !=) can be used
  to compare values of any type (i.e., int, double, or boolean).
  It is permitted to use one argument of type int and the other
  of type double.  In this situation, the argument of type int
  will be converted to a double and a floating point comparison
  operation will be performed at runtime.

LogBinArgsBoolean
  The arguments of a logical operator (\&\& or $|$$|$) must be of type
  boolean.  [And the result of using any such operator will be a
  value of type boolean.]

AssignTypes
  The type of the expression on the right hand side of an
  assignment must match the type of the variable that is named on
  the left hand side.  However, this rule should be relaxed for
  assignments involving a mixture of numeric types (i.e., an int
  on one side of the "=" symbol and a double on the other).  In
  these cases, the value that is produced by the expression on
  the right hand side of the assign will be converted to match
  the type of the variable on the left hand side.

CastInt
  The argument in an IntToDouble cast must be of type int and the
  result of the cast will be of type double.  (See below for a
  description of casts.)

CastDouble
  The argument in a DoubleToInt cast must be of type double and
  the result of the cast will be of type int.  (See below for a
  description of casts.)

FunctionDefined
  Every function that is named in a function call should have a
  definition in the program.

CallReturnType
  A function that has return type void should not be called in a
  context where a value is expected.  (i.e., void functions can
  only be called from within a statement expression.)

CallNumberOfArgs
  The number of arguments in a call to a function f must be the
  same as the number of formal parameters in the definition of f.

FormalTypeMismatch
  The type of each argument in a function call must match the type
  of the corresponding formal parameter in the definition of the
  function.  Note, however, that an expression of type int may be
  used as an argument where a formal parameter of type double is
  specified.  Similarly, an expression of type double may be used
  as an argument where a formal parameter of type int is
  specified.  In these cases, appropriate casts will be used to
  convert the actual arguments that are specified in the call
  to match the types of the corresponding formal parameters.

------------------------------------------------------------------------
Extensions with respect to Homework 3:
--------------------------------------
In reading the above list, you may have noticed a small number of
new features that were not present in the previous homework:

The double type:
----------------
The language has been extended to include a new type, double, that
represents (double precision) floating point numbers.  We will also
add support for floating point literals.  These additions require some
small additions to the code that you have seen previously:

The syntax of floating point literals is described by the following
regular expression:

$|$  $<$FLOATLIT   : ($<$NUM$>$)+ "." ($<$NUM$>$)+ $>$

The grammar for types includes an extra line (and there is an extra
value, Type.DOUBLE, that represents the new double type):

$|$ "double"  { return Type.DOUBLE; }

The grammar for expressions includes a rule for floating point
literals (and there is a new class, FloatLit, for representing
floating point literals in abstract syntax trees:

$|$ t=$<$FLOATLIT$>$ { e = new FloatLit(t.image); })

(Note that we do not attempt to parse floating point literals to
extract the corresponding numeric values; instead we just pass the
underlying lexeme---a text string, t.image---as the argument to the
FloatLit() constructor.)

Casts:
------
The abstract syntax includes two new classes, IntToDouble, and
DoubleToInt, that represent casts or conversion operators.  As the
name suggests, an IntToDouble node represents an operation that
converts an integer value in to a corresponding double.  (For example,
it might convert the integer 1 in to the double value 1.0.)
Conversely a DoubleToInt node represents an operation that will
convert a double value to a corresponding value.  (For example, it is
reasonable to assume that this operation would convert 3.1415, a value
of type double, into the int value 3.) Note, however, that we will not
specify exactly what these operations do in all cases at this time;
doing that will be necessary to describe the dynamic semantics of our
language, but our sole focus for this assignment is on static
semantics.

You might notice also that there is no concrete syntax for these case
operations in our language.  (i.e., they will not be included in any
abstract syntax tree that is constructed by the parser.) Instead,
these operations are intended to be inserted in to the abstract syntax
tree during static analysis.  For example, the following indented
output shows the abstract syntax tree that the parser will construct
for an input expression 1+2.3:

  Add
    IntLit(1)
    FloatLit(2.3)

According to ArithBinArgsNumeric, however, the combination of an
integer and a floating point argument will be handled by converting
the int on the left of the "+" in to a double (using an IntToDouble
cast) and the performing a double addition on the resulting two
arguments.  In terms of code, we could execute the following Java
statement left = new IntToDouble(left) to transform the original
abstract syntax tree so that it looks like this:

  Add double
    IntToDouble
      IntLit(1)
    FloatLit(2.3)

Note that static analysis is expected to insert an IntToDouble cast
like this at any point where an expression of type int is used in a
context where a double is expected (which includes not only arithmetic
operations like "+", but also comparisons, assignments, return
statements, and function calls.)  Similarly, static analysis is
expected to insert a DoubleToInt cast at any point where the source
program used an expression of type int where a value of type double
was required.

Type annotations:
-----------------
Our language uses the same concrete syntax for adding two int values
that it does for adding two double values.  When it comes to code
generation, however, it will be important for the compiler to be able
to distinguish between these two cases: the assembly code that is
needed to implement an integer addition is quite different to the code
that is needed for a floating point addition, for example.  For this
reason, several of the abstract syntax node classes in the ast folder
have been modified to include a new "type" attribute.  It is the
responsibility of static analysis to set these fields (which are
initialized to null when the abstract syntax nodes are first
created).  Once these attributes have been set to non-null Type
values, those types will be displayed as part of the indented
abstract syntax trees that are produced by running the sample code.
The following example shows two such annotations, one for the Assign
node and one for the Add, that have been added as a result of passing
the expression "x = 1 + 2.3" through static analysis, assuming that "x"
has been defined previously as a variable of type int:

  Assign int
    "x"
    DoubleToInt
      Add double
        IntToDouble
          IntLit(1)
        FloatLit(2.3)

Type annotations like this are required on all Assign, BinExpr, and
Print nodes.

------------------------------------------------------------------------
Supporting Materials:
---------------------
This assignment requires the set of supporting materials provided in
the files hw4-static.tar.gz and hw4-static.zip.  (The content of the
two files is the same; only the file formats differ, so use whichever
one seems most natural to you.)

After unpacking the supporting materials (tar xvf hw4-static.tar.gz
or unzip hw4-static.zip, respectively), you will find a new hw4-static
directory that contains the following:

Parser.jj -- a JavaCC parser for the language that we studied in
  Homework 3.  This parser has been extended with the features
  that were described above for adding the double type and support
  for floating point literals.  You are welcome to substitute
  your own Parser.jj if you prefer, provided that you make the
  corresponding small changes.

  As you would expect, you can obtain the Java code for the Parser
  by running the command:  javacc Parser.jj

Static.java -- a simple main program for testing the static analysis
  code.  This can be compiled using the line: javac Static.java
  Once compiled, you can run this program using commands like:
        echo "void main() { print 1; }" $|$ java Static 
  or:   java Static $<$ nameOfFileContainingTestCode

compiler/*.java -- a minimal library for error handling that includes
  one class for representing errors as objects (the Failure class),
  and one (the Handler class), that implements a simple error message
  handler.

ast/*.java -- a collection of classes, most of which correspond to
  different types of abstract syntax tree nodes.  The main exceptions
  are:

  - The IndentOutput class, which is used to support the textual
    display of abstract syntax trees using indentation.

  - The TypeEnv and FunctionEnv classes, which implement simple
    linked-list data structures used to implement mappings from
    variable names to types (for global variables, function
    parameters, and locally defined variables).  These classes are
    very similar in purpose and structure to the examples of the
    same name in the Week 8 lab (envlab).

  - The Context class, which is used during static analysis to
    provide information about the context in which a given
    expression, statement, or definition appears.  More
    specifically, Context values are used to store: a TypeEnv
    environment, called globals, for all of the globally defined
    variables in the program; a function environment, called
    functions, that maps function names to Function objects; and
    an error handler for reporting compile-time errors in given
    source program.

------------------------------------------------------------------------
Type Checking Statements:
-------------------------
Each of the classes that represents an abstract syntax node for a
particular form of statement is supposed to contain an implementation
of the method:

    public TypeEnv check(Context ctxt, TypeEnv locals)
        throws Failure 

The intention here is that, if stmt is an abstract syntax tree for a
statement, then a call stmt.check(ctxt, locals) will perform whatever
tests are necessary to ensure that the statement is valid, using the
given context for information about the surrounding program, and the
specified environment for local variables.  The return value is another
environment, extending the input environment, locals, if necessary,
with entries for any variables that are defined inside stmt.

As an example, the following code shows the implementation of check()
in the Block class:

    public TypeEnv check(Context ctxt, TypeEnv locals)
      throws Failure {
        TypeEnv inner = locals; 
        for (int i=0; i$<$body.length; i++) {
            inner = body[i].check(ctxt, inner);
        }
        return locals;  // Back to original environment outside block
    }

The code shown here performs recursive calls to check() that each of
the statements inside the block are valid.  But there are two subtle
details in here:

- First, the code keeps a pointer to the local variable environment in
  the variable inner as it moves through the statements in the block.
  This allows the code to work correctly with a block like:

    { int x = 1 + 2; print x; }

  where the local variable environment for the second statement
  includes a new environment entry for the variable x, extending
  whatever environment was in scope for the definition of x at the
  start of the block.

- Second, the environment that is returned at the end of the block
  is the same environment, locals, that was given for the start of the
  block, and not the environment, inner, that was calculated at the
  end.  This reflects the fact that any local variables that are
  introduced inside a block, go out of scope when the block ends.

As it happens, however, the supporting materials only provide
implementations for this method in a very limited set of cases: one
for Stmt, one for Block, and one for Print.  So, after studying those
examples, you will need to implement a version of the check() method
for all of the other classes that correspond to statements.

How will you know if you have provided an implementation for check()
in all of the different forms of statements?  One easy way to find out
is to rewrite the current placeholder implementation of check() in the
Stmt class as an abstract method: this requires adding "abstract" at
the start of that definition and line and replacing the method body
with a single semicolon.  Once you've done that, the Java compiler
will generate an error message for any class that requires but is
missing an implementation of check().

------------------------------------------------------------------------
Type Checking Expressions:
--------------------------
Each of the classes that represents an abstract syntax node for a
particular form of expression is supposed to contain an implementation
of the following method:

    public Type typeOf(Context ctxt, TypeEnv locals)
      throws Failure

This obviously looks a lot like the check() method for statements
except that the return value is a Type rather than a TypeEnv.  This
reflects fundamental differences between statements and expressions:

- evaluating an expression will produce a result value of some type,
  unlike statements, which do not produce values.

- executing a statement may result in an extended environment, unlike
  expressions, which cannot introduce new local variables.

For example, the following code shows the implementation for typeOf()
for integer literals, as specified in the IntLit class:

    public Type typeOf(Context ctxt, TypeEnv locals)
      throws Failure {
        return Type.INT;
    }

In this particular case, arguments to typeOf() are not used:
regardless of the context and local environment in which an integer
literal appears, we can be sure that it will always produce a value
of type int, which is represented by the value Type.INT.

On the other hand, the implementation of typeOf() in the class
LogBinExpr---which covers expressions using logical operators like
"l \&\& r" or "l $|$$|$ r"---is as follows:

    public Type typeOf(Context ctxt, TypeEnv locals)
      throws Failure {
        // Find the type of the left operand:
        Type leftType = left.typeOf(ctxt, locals);
  
        // Check that the left operand produces a value of type boolean:
        if (!leftType.equals(Type.BOOLEAN)) {
            ctxt.report(new Failure("LogBinArgsBoolean"));
        }
  
        // Find the type of the right operand:
        Type rightType = right.typeOf(ctxt, locals);
  
        // Check that the right operand produces a value of type boolean:
        if (!rightType.equals(Type.BOOLEAN)) {
            ctxt.report(new Failure("LogBinArgsBoolean"));
        }
  
        // Logical operators produce results of type boolean:
        return type=Type.BOOLEAN;
    }

The last line here captures the fact that expressions like these
always produce a boolean result.  In addition, however, the earlier
lines include recursive typeOf() calls for the left and right
arguments to ensure that each of those subexpressions is valid, as
well as tests to ensure that the results produced by those
subexpressions are themselves boolean values, as required for
arguments to \&\& and $|$$|$.  If either of the arguments does not produce a
value of the appropriate type, then the line:

   ctxt.report(new Failure("LogBinArgsBoolean"));

is used to register that the program has an error.  Note that the
"LogBinArgsBoolean" code used here is just the name from our earlier
list of static checks corresponding to the rules for type checking
expressions involving logical operators.

Once again, however, the materials that have been provided for this
assignment only include typeOf() implementations for a limited set of
cases: for Expr, Id, IntLit, IntToDouble, and LogBinExpr.  So another
part of your task in this assignment will be to study those examples
and then add appropriate implementations of typeOf() to implement type
checking for all of the different forms of abstract syntax node that
correspond to expressions.

Note that careful use of inheritance can significantly reduce the
amount of code that you will need to write here.  In the example above
it made good sense to provide one copy of the code for checking
logical boolean operators in the LogBinExpr class instead of putting
copies of the same code in each of the LAnd and LOr classes.  This
reflects the fact that the type checking rules for \&\& are exactly the
same as those for $|$$|$.  As such, it is appropriate to place common code
for these operations in LogBinExpr and then let each of LAnd and LOr
inherit this behavior automatically.

------------------------------------------------------------------------
Reporting Errors:
-----------------
In the code for LogBinExpr shown previously, errors detected in input
programs were handled by calls of the form:

   ctxt.report(new Failure(codeNameForRuleGoesHere));

This is an appropriate way to handle errors in situations where there
is an easy way to recover from the error condition.  For example,
given an expression like "1 \&\& 2",  static analysis can detect that
the left hand operand is not a boolean and report that as an error.
But instead of aborting the type checking process at that point, it
can still continue to look for errors in the right hand operand.
Detecting multiple errors in a single static analysis phase like this
is often useful in practice because it can help to reduce the number
of edit-compile cycles that are needed to get working code.

In some situations, however, there may not be an obvious way to
recover from an error condition.  The implementation of typeOf() for
identifiers (in the Id class), for example, is as follows:

    public Type typeOf(Context ctxt, TypeEnv locals)
      throws Failure {
        return ctxt.findVar(name, locals).getType();
    }

This code makes use of the following method in the Context class:
[Aside: Why is this code separated out from the typeOf() method?
Hint: This function might also come in handy when you are dealing
with the implementation of assignment statements ...]

    TypeEnv findVar(String name, TypeEnv locals)
      throws Failure {
        // Look for a definition of this variable in the local environment:
        TypeEnv te = TypeEnv.find(name, locals);

        // If there is no local definition, try the global environment:
        if (te==null) {
            te = TypeEnv.find(name, globals);
        }

        // If we still have not found a definition, then there is an error:
        if (te==null) {
            throw new Failure("VarDefined");
        }
        return te;
    }

This code attempts to find an environment entry for a particular
identifier, first by looking in the local environment, and then in the
global environment.  But what type should we return if we do not find
an environment entry?  Of course, we could make an arbitrary choice
and return Type.INT as a "guess".  This might allow static analysis to
continue searching the source program for other errors, but it could
also result in spurious and confusing error conditions.  In an
expression like "x \&\& y", for example, we would expect static analysis
to report an error if the variable x were not defined, but we would be
surprised if it also reported an error for a use of \&\& with an integer
argument.

As the code above shows, we can avoid the need to make arbitrary
choices in situations like this by throwing an exception using a
statement of the form:

   throw new Failure(codeNameForRuleGoesHere);

As a general rule, it is best to use the earlier ctxt.report(...)
pattern when possible---because that will allow your static analysis
implementation to catch more bugs in a single compiler run---but it
is always possible to use the throw ... pattern as a fallback if
there is no other choice.  Choosing between these patterns will have
no impact on grading because all of the test programs that we use for
that will be limited to a single error, but this can be an important
issue in practical systems.

------------------------------------------------------------------------
So What is the Assignment?
--------------------------
As described above, your assignment for this homework exercise is
to complete, as best as you can, the implementation of a static
analysis tool that checks input programs for each of the conditions
described at the start of this document.

This task should be accomplished by adding appropriate
implementations for the check() and typeOf() methods that were
described above.

All of the code that you write should be inserted in to the
existing classes in the ast folder.  Do NOT add any new source
files, or modify the provided Static.java file in any way; your
code may not function correctly in the testbed that is used for
grading if you do not follow these instructions.  You may choose to
use your own Parser (with the modifications described above for the
double type and floating point literals), or you may use the Parser
that was provided as a solution for HW3.  Your choice of parser
should not have any bearing on the testing framework, so long as
the entry point to the parser is a function called Top that returns
an array of Defn values, as in the original HW3 Template.jj file.

IMPORTANT: You may not be able to implement checks for all of the
static analysis properties listed at the top of this document in
the time available.

Testing:
--------
As with the previous homework, we have provided a suite of test
inputs in the tests subfolder, each of which is accompanied by a
corresponding .ref file that shows the output that is produced by
the reference solution.  Note that some of these inputs represent
valid inputs to the static analyzer, but some are are invalid
inputs.  For the former, the output in the .ref file will be an
annotated, indented display of the abstract syntax tree for the
input program.  For the latter, the output will be a suitable error
diagnostic.  Your program should replicate the behavior of the
reference solution, matching its output in each case.

The following commands may be of particular use to you during this
exercise:

- make   will automate the process of running JavaCC to build the
  parser, and of recompiling the additional Java code after you have
  made changes.

- ./runTests   will run your static analyzer on all of the provided
  tests, producing a summary of the results at the end.  Note,
  however, that you are strongly encouraged to look at and use the
  tests individually as necessary.  For example, to run test039
  through your static analyzer, type "java Static tests/test039" and
  then inspect the output to see if it agrees with what you expect.

- ./showTests  will display a summary of all the tests and your
  most recent set of results in each case.

It is possible for the static analyzer to detect multiple errors in
a program within a single run.  The provided tests, however, have
been designed so that each input program should trigger at most one
error message.

The test suite contains a total of 110 individual tests, numbered
from 000 to 109.  The code that is supplied already passes 20 of
these tests.  Ideally, your final solution should pass all of the
tests, but I do not expect that all solutions will meet this goal.
You should also be aware that the supplied test suite is not
complete.  A larger set of tests will be used for grading, and you
are welcome to develop additional tests of your own devising to
ensure that your solution is working correctly.

Although there are 90 failing tests in the supplied code, you will
often find that adding one (correct) implementation of a typeOf() or
check() method can fix multiple tests in a single step.  For this
reason, as you work through this exercise, I encourage you to focus,
not on the number of tests that you are currently passing, but
rather on the implementations of check() or typeOf() that you still
need to add or modify.

Extra credit:
-------------
There are a number of additional program properties that we might
want a static analyzer to check for in addition to those listed at
the start of this document.  You may obtain extra credit by including
a list of UP TO FIVE additional example program properties to
illustrate this point.  These examples must be written in a plain
text file called "extras.txt" that is included with your submission,
and, to qualify, each one must include:

- A brief, high-level description of what property will be checked.

- A small example program that illustrates the property.

- A sentence indicating why this check might be important:
  - Would it help to detect program errors in input programs?
  - Would it help to clarify ambiguities in the code prior to code
    generation?
  - Would it help the compiler to generate more efficient code,
    either in terms of program execution time or size?

You are not expected to implement any of the properties that you
describe here.  Note, however, that no credit will be given for
properties that either duplicate items listed at the start of this
document, or for items that we could not reasonably hope to check
during static analysis (e.g., without running the program).

------------------------------------------------------------------------
\end{document}
